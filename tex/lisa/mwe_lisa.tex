\documentclass[]{article}

%\usepackage{fontspec}
%\usepackage[iosevka]{lisa}
\usepackage[]{lisa}

%opening
\title{Minimal \lisa!lisa! Example}
\author{Richard Schmoetten}

\begin{document}

\maketitle

\section{Copy/Paste \lisa!Isabelle! listings: \lisa|A \\<Longrightarrow> B|}

You can write inline listings using the \lisa|\lisa| command: \lisa|A \<Longrightarrow> B|. This should\footnote{Automatically adjust the font size: \lisa|A \<Longrightarrow> B|. Thanks for making literals work, \lisa|bigfoot|!} also work in footnotes. It also works without the \lisa!bigfoot! package, provided you escape special characters. You do currently have to escape e.g. backslashes in \lisa!lisa! inline listings in titles.

Listing environments are used as in the \lisa!listings! package, with defaults set in \lisa!lisa.sty!. Notably, these are \lisa!\footnotesize!.
\begin{lstlisting}
(* This is a great example!? *)
lemma example: "A \<Longrightarrow> B \<longrightarrow> C"
  using (*...*) by simp
\end{lstlisting}

\end{document}
