\documentclass[12pt,notheorems,aspectratio=169,handout]{beamer} % notes, handout

\newcommand{\lisaTeX}{C:/Users/Richard/uoe-shared/tex/lisa}
%%% single-line packages %%%
%%%%%%%%%%%%%%%%%%%%%%%%%%%%
\usepackage[utf8]{inputenc}
\usepackage{standalone}
%\usepackage{babel}% For multilingual/non-English documents.
%\usepackage{csquotes}% Load if you want non-English quotes.
%\usepackage{xpatch}% For some biblatex styles, this is required. Not for the core ones.
%\usepackage[backend=biber,style=alphabetic]{biblatex}% Biblatex+Biber setup. Your document will now take a while to compile.
%\usepackage{graphicx}

%%% Isabelle Listings %%%
%%%%%%%%%%%%%%%%%%%%%%%%%
\usepackage{lisa}% includes amssymb, fancyvrb, listings, xcolor, bigfoot, ltxcmds

%%% hyperref %%%
%%%%%%%%%%%%%%%%
\usepackage{hyperref}
% This colours the hyperlinks, which is better for screen reading. COMMENT for printing/colorblind-friendliness.
\hypersetup{
    colorlinks,
    linkcolor={red!50!black},
    citecolor={blue!50!black},
    urlcolor={blue!80!black}
}

%%% AMS packages %%%
%%%%%%%%%%%%%%%%%%%%
\usepackage{physics}% includes amsmath, great for \bra and \ket etc.
\usepackage{mathtools}
\usepackage{amsthm}
% additional symbol definitions
\newcommand\restr[2]{{% we make the whole thing an ordinary symbol
  \left.\kern-\nulldelimiterspace % automatically resize the bar with \right
  #1 % the function
  \vphantom{\big|} % pretend it's a little taller at normal size - comment if not wanted
  \right|_{#2} % this is the delimiter
}}

% Environment definitions, mostly standard
\theoremstyle{plain}% default
\newtheorem{theorem}{Theorem}[section]
\newtheorem{lemma}[theorem]{Lemma}
\newtheorem{proposition}[theorem]{Proposition}
\newtheorem*{corollary}{Corollary}

\theoremstyle{definition}
\newtheorem{definition}{Definition}[section]
\newtheorem{axiom}{Axiom}[section]
\newtheorem{example}{Example}[section]
\newtheorem{exercise}[example]{Exercise}

\theoremstyle{remark}
\newtheorem*{remark}{Remark}
%\newtheorem*{note}{Note} % Notes are already defined in beamer - I think for presentor mode?

% TODO look into the cleveref package and hyperref's \autoref
%%% fancyref %%%
%%%%%%%%%%%%%%%%
\usepackage[plain]{fancyref}% tight spacing because I use abbreviated cross references, e.g. Fig. 1
{%%% old format -- short, abbreviated names, grammatical capitalisation only, tight spacing %%%
%% new prefixes
%\newcommand*{\fancyrefthmlabelprefix}{thm}
%\newcommand*{\fancyrefdeflabelprefix}{def}
%\newcommand*{\fancyrefeqnlabelprefix}{eqn}
%\newcommand*{\fancyreflinelabelprefix}{line}
% spacing used in all commands, not just the ones (re)defined below
%\renewcommand*{\fancyrefdefaultspacing}{\fancyreftightspacing}
%% new names and formats
%\newcommand*{\Frefdefname}{Def.}
%\newcommand*{\frefdefname}{def.}
%\frefformat{plain}{\fancyrefdeflabelprefix}{\frefdefname\fancyrefdefaultspacing#1}
%\Frefformat{plain}{\fancyrefdeflabelprefix}{\Frefdefname\fancyrefdefaultspacing#1}
%\newcommand*{\Frefeqnname}{Eqn.}
%\newcommand*{\frefeqnname}{eqn.}
%\frefformat{plain}{\fancyrefeqnlabelprefix}{\frefeqnname\fancyrefdefaultspacing#1}
%\Frefformat{plain}{\fancyrefeqnlabelprefix}{\Frefeqnname\fancyrefdefaultspacing#1}
%\newcommand*{\Freflinename}{Line}
%\newcommand*{\freflinename}{line}
%\frefformat{plain}{\fancyreflinelabelprefix}{\freflinename\fancyrefloosespacing#1}
%\Frefformat{plain}{\fancyreflinelabelprefix}{\Freflinename\fancyrefloosespacing#1}
%% existing name (and format?) changes (here, shortened)
%%\renewcommand*{\Frefeqname}{Eqn.}
%\renewcommand*{\Frefsecname}{Sec.}
%\renewcommand*{\Freftabname}{Tab.}
%\renewcommand*{\Freffigname}{Fig.}
}
% new prefixes
\newcommand*{\fancyrefthmlabelprefix}{thm}
\newcommand*{\fancyreflemlabelprefix}{lem}
\newcommand*{\fancyrefapplabelprefix}{app}
\newcommand*{\fancyrefdeflabelprefix}{def}
\newcommand*{\fancyrefeqnlabelprefix}{eqn}
\newcommand*{\fancyreflinelabelprefix}{line}
% spacing used in all commands, not just the ones (re)defined below
\renewcommand*{\fancyrefdefaultspacing}{\fancyrefloosespacing}
% new names and formats
\newcommand*{\Frefdefname}{Definition}
\newcommand*{\frefdefname}{definition}
\frefformat{plain}{\fancyrefdeflabelprefix}{\frefdefname\fancyrefdefaultspacing#1}
\Frefformat{plain}{\fancyrefdeflabelprefix}{\Frefdefname\fancyrefdefaultspacing#1}
\newcommand*{\Frefappname}{Appendix}
\newcommand*{\frefappname}{appendix}
\frefformat{plain}{\fancyrefapplabelprefix}{\frefappname\fancyrefdefaultspacing#1}
\Frefformat{plain}{\fancyrefapplabelprefix}{\Frefappname\fancyrefdefaultspacing#1}
\newcommand*{\Frefeqnname}{Equation}
\newcommand*{\frefeqnname}{equation}
\frefformat{plain}{\fancyrefeqnlabelprefix}{\frefeqnname\fancyrefdefaultspacing#1}
\Frefformat{plain}{\fancyrefeqnlabelprefix}{\Frefeqnname\fancyrefdefaultspacing#1}
\newcommand*{\Freflemname}{Lemma}
\newcommand*{\freflemname}{lemma}
\frefformat{plain}{\fancyreflemlabelprefix}{\freflemname\fancyrefdefaultspacing#1}
\Frefformat{plain}{\fancyreflemlabelprefix}{\Freflemname\fancyrefdefaultspacing#1}
\newcommand*{\Frefthmname}{Theorem}
\newcommand*{\frefthmname}{theorem}
\frefformat{plain}{\fancyrefthmlabelprefix}{\frefthmname\fancyrefdefaultspacing#1}
\Frefformat{plain}{\fancyrefthmlabelprefix}{\Frefthmname\fancyrefdefaultspacing#1}
\newcommand*{\Freflinename}{Line}
\newcommand*{\freflinename}{line}
\frefformat{plain}{\fancyreflinelabelprefix}{\freflinename\fancyrefloosespacing#1}
\Frefformat{plain}{\fancyreflinelabelprefix}{\Freflinename\fancyrefloosespacing#1}
% existing name (and format?) changes (here, shortened)
%\renewcommand*{\Frefeqname}{Eqn.}
\renewcommand*{\Frefsecname}{Section}
\renewcommand*{\Freftabname}{Table}
\renewcommand*{\Freffigname}{Figure}
% Capitalise everything!
\renewcommand{\fref}{\Fref}

%%% TiKZ %%%
\usepackage{tikz}
%\usetikzlibrary{arrows}
\usetikzlibrary{cd}
\usetikzlibrary{decorations.pathreplacing,calc} % needed for braces across itemize (below)
\usetikzlibrary{tikzmark} % needed for drawing rectangles around highlights in lstlistings

%%% mathtools %%%
%%%%%%%%%%%%%%%%%
\usepackage{mathtools}
\DeclarePairedDelimiter{\all}{\forall}{.\quad}
\DeclarePairedDelimiter{\any}{\exists}{.\quad}
%%% strikeout in math mode
% from : https://tex.stackexchange.com/a/20613
\newcommand\hcancel[2][black]{\setbox0=\hbox{$#2$}%
\rlap{\raisebox{.45\ht0}{\textcolor{#1}{\rule{\wd0}{1pt}}}}#2}


%%% Symbol shortcuts and custom commands %%%
%%%%%%%%%%%%%%%%%%%%%%%%%%%%%%%%%%%%%%%%%%%%
\newcommand{\setN}{{\mathord{\mathbb N}}}
\newcommand{\setZ}{{\mathord{\mathbb Z}}}
\newcommand{\setQ}{{\mathord{\mathbb Q}}}
\newcommand{\setR}{{\mathord{\mathbb R}}}
\newcommand{\setC}{{\mathord{\mathbb C}}}
\newcommand{\setH}{{\mathord{\mathbb H}}}

\def\ie/{\textit{i}.\textit{e}.}
\def\eg/{\textit{e}.\textit{g}.}
\def\cf/{\textit{cf}.}

%%% Drawing rectangles to highlight parts of listings %%%
\newcounter{lstmark}
\newcommand{\tmark}[2][0pt]{%tikzmark
    \stepcounter{lstmark}%
    \if###2##%
    \else
        #2%
    \fi
    \tikz[overlay,remember picture]\node (\thelstmark){};%
}
\newcommand{\makehl}[3][0pt]{%make highlight
    \begin{tikzpicture}[overlay, remember picture]
        \draw[red,rounded corners]
          let \p1=(#2), \p2=(#3) in
          ({\x1-3pt}, {\y1+1.65ex})
            rectangle
          ({\x2+3pt}, {\y2-0.65ex});
    \end{tikzpicture}%
}

%%% Braces over multiple itemized items %%%
\newcounter{itemnum}
\newcommand{\nt}[2][0pt]{%
    \stepcounter{itemnum}%
    \if###2##%
    \else
        #2%
        \thinspace
    \fi
    \tikz[overlay,remember picture,baseline=(\theitemnum.base),xshift=#1]\node (\theitemnum){};%
}
\newcommand{\makebrace}[4][0pt]{%
    \begin{tikzpicture}[overlay, remember picture]
        \draw [decoration={brace,amplitude=0.5em},decorate]
        let \p1=(#2), \p2=(#3) in
        ({max(\x1+#1,\x2+#1)}, {\y1+1.75ex}) --
            node[right=0.6em] {#4} ({max(\x1+#1,\x2+#1)}, {\y2-0.5ex});
    \end{tikzpicture}%
}
\newenvironment{braceitems}{%
    \begin{enumerate}
}{%
    \end{enumerate}
    \setcounter{itemnum}{0}%
}


%%% single-line packages %%%
%%%%%%%%%%%%%%%%%%%%%%%%%%%%
\usepackage{setspace}% For adjusting spacing between lines\comment{text}{comment}

%%% Custom commands %%%
%%%%%%%%%%%%%%%%%%%%%%%

% Borderless frame for including big images
% TODO not well tested - may break some title bars or footers?
% TODO may also benefit from some more refinement over just setting all margins to 0
\newenvironment{emptyframe}
{
 % not too sure, but may be needed if you have a background image
 % that should not appear on this kind of frame:
 \setbeamertemplate{background canvas}[default]
 % turn off navigation symbols for this frame
 \setbeamertemplate{navigation symbols}{}
 % locally set margins to zero: (notice the use of \bgroup ... \egroup
 % to limit the scope of the geometry restriction
 % where curly brackets {} aren't possible)
 \bgroup \newgeometry{margin=0cm}
 \begin{frame}[plain]
}
{
 \end{frame}
 \egroup
}

% footnote without label, e.g. for same-slide references
\newcommand{\nolabelfootnote}[1]{%
  \begingroup
  \renewcommand{\thefootnote}{}
  \footnote[frame]{\hspace{-18pt}\tiny#1}%
  \addtocounter{footnote}{-1}%
  \endgroup
}

% Improved version of \alt. Keeps constant width to avoid jiggling slides.
% https://tex.stackexchange.com/questions/13793/beamer-alt-command-like-visible-instead-of-like-only
\makeatletter
% Detect mode. mathpalette is used to detect the used math style
\newcommand<>\Alt[2]{%
    \begingroup
    \ifmmode
        \expandafter\mathpalette
        \expandafter\math@Alt
    \else
        \expandafter\make@Alt
    \fi
    {{#1}{#2}{#3}}%
    \endgroup
}
% Un-brace the second argument (required because \mathpalette reads the three arguments as one
\newcommand\math@Alt[2]{\math@@Alt{#1}#2}
% Set the two arguments in boxes. The math style is given by #1. \m@th sets \mathsurround to 0.
\newcommand\math@@Alt[3]{%
    \setbox\z@ \hbox{$\m@th #1{#2}$}%
    \setbox\@ne\hbox{$\m@th #1{#3}$}%
    \@Alt
}
% Un-brace the argument
\newcommand\make@Alt[1]{\make@@Alt#1}
% Set the two arguments into normal boxes
\newcommand\make@@Alt[2]{%
    \sbox\z@ {#1}%
    \sbox\@ne{#2}%
    \@Alt
}
% Place one of the two boxes using \rlap and place a \phantom box with the maximum of the two boxes
\newcommand\@Alt[1]{%
    \alt#1%
        {\rlap{\usebox0}}%
        {\rlap{\usebox1}}%
    \setbox\tw@\null
    \ht\tw@\ifnum\ht\z@>\ht\@ne\ht\z@\else\ht\@ne\fi
    \dp\tw@\ifnum\dp\z@>\dp\@ne\dp\z@\else\dp\@ne\fi
    \wd\tw@\ifnum\wd\z@>\wd\@ne\wd\z@\else\wd\@ne\fi
    \box\tw@
}
\makeatother


\usepackage[backend=biber,style=alphabetic]{biblatex}
\addbibresource{C:/Users/Richard/library/refs-biblatex.bib}

%\usepackage{todonotes}
\usepackage{pgfgantt}

\usetikzlibrary{mindmap,fit,backgrounds}
\usetikzlibrary{decorations.pathreplacing,calc}

%\newcommand{\M}{\ensuremath{\mathbb M}}
%\newcommand{\A}{\ensuremath{\mathcal A}}
\newcommand{\id}{\ensuremath{\text{id}}}
%\newcommand{\T}{\ensuremath{\mathcal T}}
%\renewcommand{\P}{\ensuremath{\mathcal{P}_0}}
%\renewcommand{\L}{\ensuremath{\mathcal L}}
%\renewcommand{\O}{\ensuremath{\mathcal O}}
\newcommand{\Diff}{\ensuremath{\mathrm{Diff}}}
\newcommand{\GL}{\ensuremath{\text{GL}}}
\newcommand{\SL}{\ensuremath{\text{SL}}}

\newcommand{\orange}{\color{orange}}
\newcommand{\red}{\color{red}}
\newcommand{\green}{\color{green}}

\hypersetup{linkcolor=black}


%
% Choose how your presentation looks.
%
% For more themes, color themes and font themes, see:
% http://deic.uab.es/~iblanes/beamer_gallery/index_by_theme.html
%
\mode<presentation>
{
  \usetheme{SimplePlus}      % or try Darmstadt, Madrid, Warsaw, ...
%   \usecolortheme{default} % or try albatross, beaver, crane, beetle, seagull, dove, fly ...
  \usefonttheme{default}  % or try serif, structurebold, ...
  \setbeamertemplate{navigation symbols}{}
  \setbeamertemplate{caption}[numbered]
}

%\usepackage[english]{babel}
%\usepackage[utf8x]{inputenc}
%\usepackage[]{amsmath, amsthm}
%\usepackage{cite}
%\usepackage{standalone}
%\usepackage[labelformat=empty, font=small]{caption}
%\usepackage{tikz}
%\usepackage{xcolor}
%\usetikzlibrary{arrows}

\title[]{%
  From Lie algebras to Frobenius's Theorem\\%
  \normalsize{Algebraic Formalisation with locales, types and relations}}
\author{Richard Schmoetten}
\institute{The University of Edinburgh}
\date{27th June 2023}

\begin{document}

\definecolor{grey_background}{RGB}{240,240,240}
\definecolor{grey_primary}{RGB}{90,90,90}
\definecolor{grey_secondary}{RGB}{189,189,189}
\definecolor{grey_3}{RGB}{140,140,140}

%\setbeamercolor{background canvas}{bg=grey_background}
%%\setbeamercolor{background}{bg=grey_background}
%\setbeamercolor{palette primary}{bg=grey_secondary,fg=grey_primary}
%\setbeamercolor{palette secondary}{bg=grey_secondary,fg=grey_primary}
%\setbeamercolor{palette tertiary}{bg=grey_secondary,fg=grey_primary}
%\setbeamercolor{palette quaternary}{bg=grey_secondary,fg=grey_primary}
%\setbeamercolor{structure}{bg=grey_secondary, fg=grey_primary} % itemize, enumerate, etc
%\setbeamercolor{section in toc}{bg=grey_background} % TOC sections

%\setbeamertemplate{footline}[frame number] % numbering
%\setbeamertemplate{footline}[text line]{%
%  \parbox{\linewidth}{\vspace*{-8pt}Formalising AQFT\hfill\hspace{-50pt}\insertshortauthor\hfill\insertframenumber}%
%}% for navigation symbols with this footer: https://tex.stackexchange.com/questions/105613/footer-in-beamer

% % Override palette coloring with secondary
% \setbeamercolor{subsection in head/foot}{bg=grey_background,fg=white}

%\setbeamerfont{frametitle}{size=16pt}

\begin{frame}
  \titlepage
\end{frame}
\note{}


%%%%%%%%%%%%%%
%%% Part 1 %%%
%%%%%%%%%%%%%%

\begin{frame}[fragile]\frametitle{Lie Groups and Algebras}
\note[item]{We've talked about groups and manifolds: a Lie group is both.}
\begin{overlayarea}{\textwidth}{.4\textheight}
\begin{itemize}[< +- >]
\item<1- > A Lie group is a manifold that is also a group under a smooth operation $(x,y) \mapsto xy$ with smooth inverses $x \mapsto x^{-1}$.
\item<1- > A Lie algebra is anticommutative and obeys the Jacobi identity:
  \[\forall x,y,z \in \mathfrak g: \qquad [x, [y, z]] + [y, [z, x]] + [z, [x, y]] = 0\]
\item<1- |handout:0> The only obvious link between these definitions is their name...
\item<4- |handout:0> ... but we can (will) prove that every Lie group gives rise to a Lie algebra.
\end{itemize}
\note[item]<3- >{Correspondences between Lie groups and Lie algebras are the main draw of the theory: some questions that are hard in calculus are easier in algebra.}
\end{overlayarea}

\begin{overlayarea}{\textwidth}{.5\textheight}
\begin{onlyenv}<1>
\begin{lstlisting}
locale lie_grp =
    c_manifold charts \<infinity> +
    grp_on carrier tms one +
  assumes smooth_mult: "diff_on_product_manifold charts tms"
      and smooth_inv: "diff \<infinity> charts charts invs"
\end{lstlisting}
\end{onlyenv}
\begin{onlyenv}<2|handout:0>
\begin{lstlisting}
locale lie_algebra =
    algebra_on \<gg> scale lie_bracket +
    alternating_bilinear_on \<gg> scale lie_bracket +
  assumes "\<lbrakk>x\<in>\<gg>; y\<in>\<gg>; z\<in>\<gg>\<rbrakk> \<Longrightarrow> jacobi_identity lie_bracket x y z"
\end{lstlisting}
\end{onlyenv}
\end{overlayarea}

\end{frame}


%%%%%%%%%%%%%%
%%% Part 2 %%%
%%%%%%%%%%%%%%

\begin{frame}[fragile]\frametitle{Type Classes: Polymorphism in Isabelle/HOL}
\note{.}
\begin{onlyenv}<1-2|handout:0>
\phantom{Quantification over types is not possible in HOL.}
\end{onlyenv}
\begin{onlyenv}<3- >
Quantification over types is not possible in HOL.
\end{onlyenv}
\[
\Alt<1|handout:0>
  {\all{\alpha} \exists a \in (\mathrm{UNIV} :: \alpha \:\mathrm{set})}
  {\Alt<2|handout:0>
    {\all{\alpha} P (\alpha) \implies \exists a \in (\mathrm{UNIV} :: \alpha \:\mathrm{set})}
    {\hcancel[red]{\all{\alpha} P (\alpha) \implies \exists a \in (\mathrm{UNIV} :: \alpha \:\mathrm{set})}}
  }
\]
\pause
\pause
\pause
Type classes offer a (restricted) alternative. A class can be defined much like a locale: it is possible that no types satisfy the sort constraints.
\begin{lstlisting}
class fin_dim_real_vector = real_vector +
  fixes basis
  assumes finite_Basis: "finite basis"
  and independent_Basis:
    "\<nexists>u. (\<Sum>v\<in>basis. u v *\<^sub>R v) = 0 \<and> (\<exists>v\<in>basis. u v \<noteq> 0)"
  and span_Basis: "{\<Sum>a\<in>basis. r a *\<^sub>R a |r. True} = UNIV"
\end{lstlisting}
\end{frame}



\begin{frame}[fragile]\frametitle{Transfer: Relators}
Relators allow us to build transfer rules for constants from transfer relations. The most ubiquitous one is the function relator.
\pause

\vfill
\begin{minipage}{0.2\textwidth}
\begin{figure}
  \centering
  \begin{tikzcd}
    \alpha \arrow[d, "f"'] \arrow[r, "\cong"] & \beta \arrow[d, "g"] \\
    \alpha \arrow[r, "\cong"']                & \beta
  \end{tikzcd}
\end{figure}
\end{minipage}\hfill%
\begin{minipage}{0.7\textwidth}
\begin{itemize}[< +- >]
  \item $\alpha$ and $\beta$ \emph{related} by
    $\;\cong \colon \alpha \to \beta \to \texttt{bool}$
  \item Say a theorem involves a function $f\colon \alpha \to \alpha$, and we have a candidate $g\colon \beta \to \beta$ with equivalent behaviour
  \item Isabelle can use a transfer rule \Alt<4>{}{constructed from $(\cong)$ and the function relator:}
    \[\Alt<4>
      {\all{a,b} a \cong b \implies f(a) \cong g(b)}
      {[(\cong)===>(\cong)] \; f \; g}
    \]
\end{itemize}
  \note[item]<4- >{And then it works! It works both ways too.}
  \note[item]<5- >{This is the great contribution of transfer: generic construction for rules, and automation.}
  \note[item]{}
\end{minipage}
\end{frame}



\begin{frame}[fragile]\frametitle{(Real) Division Algebras}
A (associative) division algebra is an (associative) algebra that has multiplicative identity and inverses.
\pause

\begin{overlayarea}{\textwidth}{0.25\textheight}
\begin{onlyenv}<2|handout:0>
\begin{lstlisting}
locale assoc_div_algebra_on =
  assoc_algebra_1_on S scale (\<Zspot>) \<one> +
  div_algebra_on S scale (\<Zspot>) (*...*)
\end{lstlisting}
\end{onlyenv}

\begin{onlyenv}<3- >
\begin{lstlisting}
class scalar_algebra = real_div_algebra + fin_dim_real_vector
\end{lstlisting}
\end{onlyenv}
\end{overlayarea}
\pause[4]

\begin{minipage}{.47\textwidth}
\begin{example}
\begin{itemize}
\item<4- > $\setR$ is a division algebra over itself.
\item<5- > $\setC$ is a division algebra over $\setR$.
  \note[item]{You can scale complex numbers by real numbers, and you can add and multiply them.}
\item<6- > $\setH$ is a division algebra over $\setR$.
\end{itemize}
\end{example}
\end{minipage}%
\hfill%
\begin{minipage}{.53\textwidth}
\begin{itemize}
\item[]<6- > $\setH = \left\lbrace a+bi+cj+dk \;|\; a,b,c,d \in \setR \right\rbrace$
\item[]<6- > $i^2 = j^2 = k^2 = ijk = -1$
\item[]<6- > not commutative: $ij = k$ but $ji = -k$
\end{itemize}
\end{minipage}
\end{frame}



\begin{frame}[fragile]\frametitle{Polynomials over Type Embeddings}
\note{.}
The fundamental theorem of algebra states that every complex polynomial has a root. This implies every complex polynomial can be factorised.
$$p_\setC(x) = c \prod_{j=0}^{\mathrm{deg}(p)}\Big(x - r(j)\Big)$$

\begin{lstlisting}
definition "c \<simeq> r \<equiv> c = (r *\<^sub>R 1)"
definition "p \<doteq> q \<equiv> (\<forall>i::nat. coeff p i \<simeq> coeff q i)"
\end{lstlisting}

$$p_\setR(x) = r \prod_{j=0}^{N_r}\Big(x - r_r(j)\Big) \prod_{k=0}^{N_i}\Big(x - r_i(k)\Big)\Big(x - \overline{r_i(k)}\Big)$$
\end{frame}



\end{document}