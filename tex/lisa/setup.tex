%%% single-line packages %%%
%%%%%%%%%%%%%%%%%%%%%%%%%%%%
\usepackage{xcolor}
\usepackage[top=2cm,bottom=2cm,left=2cm,right=4cm]{geometry}
\usepackage{standalone}
%\usepackage{lisa}

%%% hyperref %%%
%%%%%%%%%%%%%%%%
\usepackage{hyperref}
% This colours the hyperlinks, which is better for screen reading. COMMENT for printing/colorblind-friendliness.
\hypersetup{
    colorlinks,
    linkcolor={red!50!black},
    citecolor={blue!50!black},
    urlcolor={blue!80!black}
}

%%% AMS packages %%%
%%%%%%%%%%%%%%%%%%%%
\usepackage{amsmath, amsthm, amssymb}
\theoremstyle{definition}
\newtheorem{definition}{Definition}[section]
\theoremstyle{remark}
\newtheorem*{rem}{Remark}
\newtheorem*{note}{Note}

%%% fancyref %%%
%%%%%%%%%%%%%%%%
\usepackage[plain]{fancyref}% tight spacing because I use abbreviated cross references, e.g. Fig. 1
\renewcommand*{\fancyrefdefaultspacing}{\fancyreftightspacing}
% existing name and (format?) changes (here, shortened)
\renewcommand*{\Frefeqname}{Eqn.}
% new prefixes
\newcommand*{\fancyrefthmlabelprefix}{thm}
\newcommand*{\fancyrefdeflabelprefix}{def}
\newcommand*{\fancyrefeqnlabelprefix}{eqn}
% new names and formats
\newcommand*{\Frefdefname}{Def.}
\newcommand*{\frefdefname}{def.}
\frefformat{plain}{\fancyrefdeflabelprefix}{\frefdefname\fancyrefdefaultspacing#1}
\Frefformat{plain}{\fancyrefdeflabelprefix}{\Frefdefname\fancyrefdefaultspacing#1}
\newcommand*{\Frefeqnname}{Eqn.}
\newcommand*{\frefeqnname}{eqn.}
\frefformat{plain}{\fancyrefeqnlabelprefix}{\frefeqnname\fancyrefdefaultspacing#1}
\Frefformat{plain}{\fancyrefeqnlabelprefix}{\Frefeqnname\fancyrefdefaultspacing#1}

%%% TiKZ %%%
\usepackage{tikz}
%\usetikzlibrary{arrows}

%%% mathtools %%%
%%%%%%%%%%%%%%%%%
\usepackage{mathtools}
\DeclarePairedDelimiter\norm\lVert\rVert

\newcommand{\setR}{\ensuremath{\mathbb R}}
\newcommand{\setC}{\ensuremath{\mathbb C}}
\newcommand{\clause}[2]{\ensuremath{#1 .\quad #2}}